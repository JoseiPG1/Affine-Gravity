\documentclass[10pt,a4paper]{article}
\usepackage[utf8]{inputenc}
\usepackage{amsmath}
\usepackage{amsfonts}
\usepackage{amssymb}

\usepackage{vmargin}

\usepackage{slashed}


\title{Affine Inflation in Polynomial Affine Gravity in $3+1$ dimensions with torsion}
\author{Jose Perdiguero Garate}
\date{20/12/2022}			

\begin{document}

\maketitle

\begin{abstract}
  The Polynomial Affine Gravity its a purely affine model that mediates gravitational interactions solely and exclusive through the
  affine connection instead of the metric tensor. In this paper we study solutions to the field equations including the presence of the torsion.
  Then, we study the cosmological consequence of using the Ricci tensor and Poplaswki's torison-metric tensor under the cosmological solutions.
\end{abstract}

\tableofcontents

\section{Introduction}


\section{Polynomial Affine Gravity}

The Polynomial Affine Gravity model its a purely affine model on which we endowed the manifold only with an affine connection 
$(\mathcal{M}, \Gamma)$. This allow us to define the notion of parallelism by the covariant derivative $\nabla$. Since we only
have an affine connection $\Gamma$ we can only deffine the following chain of geometric objects

\begin{equation}
  \Gamma_{\mu}{}^{\sigma}{}_{\nu} \to
  \nabla_\mu \to \mathcal{R}_{\mu\sigma}{}^{\tau}{}_{\nu} \to \mathcal{R}_{\mu\nu}  
\end{equation} 

Notice that in the absence of the metric tensor it is not possible to define the $\mathcal{R}$.

In order to built the action of the Polynomial Affine Gravity we use the irreducible fields of the affine connection,
by separating the connection into its symmetric and antisymmetric part

\begin{equation}
  \hat{\Gamma}_{\mu}{}^{\sigma}{}_{\nu} = \Gamma_{\mu}{}^{\sigma}{}_{\nu} + \mathcal{B}_{\mu}{}^{\sigma}{}_{\nu} + \delta^{\sigma}_{[\mu}\mathcal{A}_{\nu]}
\end{equation}

where $\Gamma_{\mu}{}^{\sigma}{}_{\nu}$ correspond to the symmetric part of the connection, $\mathcal{B}_{\mu}{}^{\sigma}{}_{\nu}$ its the traceless
part of the torsion tensor and $\mathcal{A}_{\mu}$ its the vectorial part of the torsion tensor. Additionally, we need to define the volume
form, which can be written using only the wedge product

\begin{equation}
  \mathrm{d}V^{\alpha\beta\gamma\delta} = J(x)\mathrm{d}x^{\alpha}\wedge\mathrm{d}x^{\beta}\wedge\mathrm{d}x^{\gamma}\wedge\mathrm{d}x^{\delta}
\end{equation}

The action must preserv the invariance under diffemorphism, which is why the symmetric part of the connection goes indirectly throught the 
covariant derivative. The fundamental fields to build the action are $\nabla, \mathcal{A}, \mathcal{B}, \mathrm{d}V$. Then we perform a sort
of \textit{dimensional structural analysis technique} studying everysingle possible non-trivial contribution to the action.

THen, the most general action in $3+1$ dimension up to boundary terms is
\begin{equation*}
    \begin{split}
      S
      & =
      \int  \mathrm{d}V^{\alpha \beta \gamma \delta} \bigg[
      B_1 \mathcal{R}_{\mu\nu}{}^{\mu}{}_{\rho}\mathcal{B}_{\alpha}{}^{\nu}{}_{\beta}\mathcal{B}_{\gamma}{}^{\rho}{}_{\delta}
      + B_2 \mathcal{R}_{\alpha\beta}{}^{\mu}{}_{\rho} \mathcal{B}_{\gamma}{}^{\nu}{}_{\delta} \mathcal{B}_{\mu}{}^{\rho}{}_{\nu}
      + B_3 \mathcal{R}_{\mu\nu}{}^{\mu}{}_{\alpha} \mathcal{B}_{\beta}{}^{\nu}{}_{\gamma} \mathcal{A}_\delta
      + B_4 \mathcal{R}_{\alpha\beta}{}^{\sigma}{}_{\rho}\mathcal{B}_{\gamma}{}^{\rho}{}_{\delta}\mathcal{A}_\sigma
      \\
      & \quad
      + B_5 \mathcal{R}_{\alpha \beta}{}^{\rho}{}_{\rho} \mathcal{B}_{\gamma}{}^{\sigma}{}_{\delta} \mathcal{A}_\sigma
      + C_1 \mathcal{R}_{\mu\alpha}{}^{\mu}{}_{\nu} \nabla_\beta \mathcal{B}_{\gamma}{}^{\nu}{}_{\delta}
      + C_2 \mathcal{R}_{\alpha\beta}{}^{\rho}{}_{\rho} \nabla_\sigma \mathcal{B}_{\gamma}{}^{\sigma}{}_{\delta}
      + D_1 \mathcal{B}_{\nu}{}^{\mu}{}_{\lambda} \mathcal{B}_{\mu}{}^{\nu}{}_{\alpha} \nabla_\beta \mathcal{R}_{\gamma}{}^{\lambda}{}_{\delta}
      \\
      & \quad
      + D_2 \mathcal{B}_{\alpha}{}^{\mu}{}_{\beta} \mathcal{B}_{\mu}{}^{\lambda}{}_{\nu} \nabla_{\lambda} \mathcal{B}_{\gamma}{}^{\nu}{}_{\delta}
      + D_3 \mathcal{B}_{\alpha}{}^{\mu}{}_{\nu}\mathcal{B}_{\beta}{}^{\lambda}{}_{\gamma} \nabla_\lambda \mathcal{B}_{\mu}{}^{\nu}{}_{\delta}
      + D_4 \mathcal{B}_{\alpha}{}^{\lambda}{}_{\beta}\mathcal{B}_{\gamma}{}^{\sigma}{}_{\delta}\nabla_\lambda \mathcal{A}_\sigma
      + D_5 \mathcal{B}_{\alpha}{}^{\lambda}{}_{\beta} \mathcal{A}_\sigma \nabla_\lambda \mathcal{B}_{\gamma}{}^{\sigma}{}_{\delta}
      \\
      & \quad
      + D_6 \mathcal{B}_{\alpha}{}^{\lambda}{}_{\beta}\mathcal{A}_\gamma \nabla_\lambda A_\delta
      + D_7\mathcal{B}_{\alpha}{}^{\lambda}{}_{\beta} \mathcal{A}_\lambda \nabla_\gamma A_\delta
      + E_1\nabla_\rho \mathcal{B}_{\alpha}{}^{\rho}{}_{\beta} \nabla_\sigma \mathcal{B}_{\gamma}{}^{\sigma}{}_{\delta}
      + E_2 \nabla_\rho \mathcal{B}_{\alpha}{}^{\rho}{}_{\beta} \nabla_\gamma \mathcal{A}_\delta
      \\
      &\quad
      + F_1 \mathcal{B}_{\alpha}{}^{\mu}{}_{\beta} \mathcal{B}_{\gamma}{}^{\sigma}{}_{\delta} \mathcal{B}_{\mu}{}^{\lambda}{}_{\rho} \mathcal{B}_{\sigma}{}^{\rho}{}_{\lambda}
      + F_2\mathcal{B}_{\alpha}{}^{\mu}{}_{\beta} \mathcal{B}_{\gamma}{}^{\nu}{}_{\lambda} \mathcal{B}_{\delta}{}^{\lambda}{}_{\rho} \mathcal{B}_{\mu}{}^{\rho}{}_{\nu}
      + F_3 \mathcal{B}_{\nu}{}^{\mu}{}_{\lambda} \mathcal{B}_{\mu}{}^{\nu}{}_{\alpha} \mathcal{B}_{\beta}{}^{\lambda}{}_{\gamma} \mathcal{A}_\delta
      + F_4 \mathcal{B}_{\alpha}{}^{\mu}{}_{\beta}\mathcal{B}_{\gamma}{}^{\nu}{}_{\delta}\mathcal{A}_\mu \mathcal{A}_\nu \bigg].
    \end{split}
  \end{equation*}

In previous works we have mentioned some of the features of the action: (i) Its rigidity, since contains all possible combinations of the fields and their derivatives; 
(ii) All the coupling constants are dimensionless, which might be a sign of conformal invariance, and also ensure that the model is power-counting renormalisable; 
(iii) The field equations are second order differential equations, and the Einstein spaces are a subset of their solu- tions; 
(iv) The supporting symmetry group is the group of diffeomorphisms, desirable for the background independence of the model; 
(v) Even though there is no fundamental metric, it is possible to obtain emergent (connection- descendent) metric tensors; 
(vi) The cosmological constant appears in the solutions as an integration constant, changing the paradigm concerning its interpretation; 
(vii) The model can be extended to be coupled with a scalar field, and the field equations are equivalent to those of General Relativity interacting with a massless scalar field. 

The field equations are obtained using Kiwosjki’s formalism and taking into account the symmetries and properties of the fundamental fields, 
we vary the action with respect to the fundamental fields which are $\Gamma$, $\mathcal{B}$ and $\mathcal{A}$. The field equations are presented in Appendix C.

In order to solve the field equations, one need to build an ansatz, since we want to do cosmology, we need to build an ansazts
compatible with the symmetries of the cosmological principle, which are rotation and translations. It is possible to build an ansatz for our
fundamental geometric objects using the Lie derivative along the Killing vector fields. The most general ansatz for the symmetric part of the 
connection $\Gamma_{\mu}{}^{\sigma}{}_{\nu}$ is
\begin{align}
      \Gamma_{t}{}^{t}{ }_{t} & =f(t), \quad \Gamma_{i}{ }^{t}{ }_{j}=g(t) S_{i j} \\
      \Gamma_{t}{ }^{i}{ }_{j} &= h(t) \delta^{i}_{j}, \quad \Gamma_{i}{ }^{j}{ }_{k}= \gamma_{i}{ }^{j}{ }_{k}
\end{align}
Notice that $f(t)$ affine function can vanishes completely by a time reparametrization. The same procedure applies to build the ansatz for the torsion tensor. The traceless part of the torsion 
tensor $\mathcal{B}_{\mu}{}^{\sigma}{}_{\nu}$ and its vectorial part $\mathcal{A}_\mu$ is completely define by only one time depending function
\begin{align*}
    \mathcal{B}_{\theta}{ }^{r}{ }_{\varphi} & = \psi (t) r^2\sin\theta \sqrt{1 - \kappa r^2} &
    \mathcal{B}_{r}{}^{\theta}{}_{\varphi} & =\frac{\psi (t) \sin \theta}{\sqrt{1 - \kappa r^2}} \\
    \mathcal{B}_{r}{}^{\varphi}{}_{\theta} & =\frac{\psi(t)}{ \sqrt{1-\kappa r^{2}} \sin \theta} & \mathcal{A}_{t} & = \eta(t)
\end{align*}

Under the cosmological ansatz the field equations can be written as
\begin{align*}
  \left(B_3(2 \kappa+g h+\dot{g})+2 B_4(g h-\dot{g})+2 D_6 \eta g-2 F_3 \psi^2\right) \psi & =0 \\
  \left(B_3 \eta \psi-2 B_4 \eta \psi-C_1(2 \eta \psi-\dot{\psi})\right) g & = 0 \\
  \left(B_3+2 B_4\right) \eta g \psi+2 C_1(\kappa \psi+4 g h \psi-g \dot{\psi}-\psi \dot{g})+2 \psi^3D & = 0 \\
  B_3((h \psi-\dot{\psi}) \eta-\psi \dot{\eta})-2 B_4((-h \psi-\dot{\psi}) \eta-\psi \dot{\eta})+C_1\left(4 h^2 \psi+2 \psi \dot{h}-\ddot{\psi}\right)+D_6 \eta^2 \psi & =0 \\
  B_3(2 \kappa+g h+\dot{g}) \eta+2 B_4(g h-\dot{g}) \eta+C_1\left(2 \kappa h+4 g h^2+2 g \dot{h}-\ddot{g}\right)+6 h \psi^2D+D_6 \eta^2 g-6 F_3 \eta \psi^2 & =0
\end{align*}
where $D = -D_1+2 D_2-D_3$.


\section{Cosmological solutions without torsion}

The field equation its given by 
\begin{equation}
  \nabla_{[\alpha}\mathcal{R}_{\beta]\gamma} = 0
\end{equation}
from here we distinguish three types of families solutions
\begin{align}
  \mathcal{R}_{\beta\gamma} & = 0 & \nabla_{\alpha}\mathcal{R}_{\beta\gamma} & = 0 & \nabla_{[\alpha]}\mathcal{R}_{\beta]\gamma} = 0
\end{align}
Here we briefly review all the possible solutions to the field equations under the cosmological ansatz.

The first type of solution requires that the Ricci tensor vanishes completly meaning $\mathcal{R}_{\beta\gamma} = 0$, this lead to two
differential equations
\begin{align}
  \dot{h} + h^2 & = 0 & \dot{g} + gh + 2\kappa & = 0
\end{align}
The two equations can be solved exactly as follow
\begin{align}
  h(t) & = \frac{1}{t - h_0} & g(t) & =  \frac{t\kappa\left(2h_0 - t\right) - g_0}{t - h_0}
\end{align}
Notice that in this scenario we do not have an emergent metric tensor.


The second type of solution requires that Ricci's covariant derivative vanishes $\nabla_{\alpha}\mathcal{R}_{\beta\gamma}  = 0$. This lead to
three differential equations system
\begin{align}
  \ddot{h} + 2h\dot{h} & = 0 \\
  2gh^2 - 2\kappa h - h\dot{g} + 3g\dot{h} & = 0 \\
  2gh^2 + 4\kappa h + h\dot{g} - g\dot{h} - \ddot{g} & = 0
\end{align}
The system of differential equations can be solve analytically by the functions
\begin{align}
  h(t) & = \sqrt{c_0}\tanh\left(\sqrt{c_0}t\right) & g(t) & = \frac{\kappa \sinh{2t\sqrt{c_0}}}{2\sqrt{c_0}} 
\end{align}
Notice that for a flat space-time requirement $\kappa = 0$, then the $g(t)$ affine functions vanishes, while the $h(t)$ remains un changed, moreover,
the spatial part of the Ricci tensor vanishes, therefore the emergent metric its degenerate. However, if $\kappa \neq 0$ then the solution does 
not vanishes the Ricci tensor, meaning that in this case we do have an emergent metric tensor.


The third type of solution requires that $\nabla_{[\alpha}\mathcal{R}_{\beta]\gamma}  = 0$, this lead to just one differential equation
\begin{align}
  4gh^2 + 2\kappa h + 2g\dot{h} - \ddot{g} = 0
\end{align}
Notice that here, we have one differential equation for the two affine functions, we can not find analytically expressions for $h(t)$ and $g(t)$ functions.



\section{Cosmological solutions with torsion}

Here we solved the field equations under three different approuches, the first one consider a straightforward approuch as a solution, the 
second method consist of using the emergent metric tensor and demand that its covariant derivative vanishes, while the third type of solution
study possible configurations of the coupling constant to obtain solutions.
\begin{align*}
  0 & = \left(B_3(2 \kappa+g h+\dot{g})+2 B_4(g h-\dot{g})+2 D_6 \eta g-2 F_3 \psi^2\right) \psi  \\
  0 & = \left(B_3 \eta \psi-2 B_4 \eta \psi-C_1(2 \eta \psi-\dot{\psi})\right) g \\
  0 & = \left(B_3+2 B_4\right) \eta g \psi+2 C_1(\kappa \psi+4 g h \psi-g \dot{\psi}-\psi \dot{g})+2 \psi^3D \\
  0 & = B_3((h \psi-\dot{\psi}) \eta-\psi \dot{\eta})-2 B_4((-h \psi-\dot{\psi}) \eta-\psi \dot{\eta})+C_1\left(4 h^2 \psi+2 \psi \dot{h}-\ddot{\psi}\right)+D_6 \eta^2 \psi  \\
  0 & = B_3(2 \kappa+g h+\dot{g}) \eta+2 B_4(g h-\dot{g}) \eta+C_1\left(2 \kappa h+4 g h^2+2 g \dot{h}-\ddot{g}\right)+6 h \psi^2D+D_6 \eta^2 g-6 F_3 \eta \psi^2 
\end{align*}
where $D = -D_1+2 D_2-D_3$.

\subsection{Logical scheme solutions}

Notice that the first two equations can be written in the following manner
\begin{align}
  0 & = A \psi  \\
  0 & = B g 
\end{align}
where $A = B_3(2 \kappa+g h+\dot{g})+2 B_4(g h-\dot{g})+2 D_6 \eta g-2 F_3 \psi^2$ and $B = B_3 \eta \psi-2 B_4 \eta \psi-C_1(2 \eta \psi-\dot{\psi})$, therefore,
we can obtain four different types of solutions
\begin{enumerate}
  \item $A = 0$ and $B = 0$
  \item $A = 0$ and $g = 0$
  \item $\psi = 0$ and $B = 0$
  \item $\psi = 0$ and $g = 0$
\end{enumerate}


\subsection{Metricity solutions}

The idea here its to constraint the emergent metric tensor by demanding that its covariant derivative vanishes, just like in general relativity. Since we have
two emergent metric tensor, here we study the two scenarios $\nabla_\alpha \mathcal{R}_{\beta\gamma} = 0$ and $\nabla_\alpha \mathcal{T}_{\beta\gamma} = 0$.


\subsection{Special configuration solution}

The idea here its to extract new type of solution that can not be obtained by previous method, here we played with different coupling constant
configurations that allow to solve the field equation.


\section{Conclusions}

Some key points about this work

\begin{itemize}
  \item Polynomial Affine Gravity formulation.
  \item Briefly review of the torsion-free solutions.
  \item Extensive analysis of field equations with torsion.
\end{itemize}


\section{Appendix A: Dimensional Analysis}

In this section we briefly show how to build the action and coupling the scalar field in the absence of the metric tensor, by using a sort of
\textit{dimensional analysis} technique.

In order to build the most general action while preserving the invariance under diffeormophism, we perform a \textit{dimensional analysis} technique. First,
we define an operator $\mathcal{N}$ to count the number of free index  and a second operator $\mathcal{W}$ to define the weight density of the object. Applying both
operators to the fundamental fields leads to
\begin{align}
  \mathcal{N}(\mathcal{A}_\mu)& = -1 & \mathcal{N}(\mathcal{B}_{\mu}{}^{\lambda}{}_{\nu})& = -1 & \mathcal{N}(\Gamma_{\mu}{}^{\lambda}{}_{\nu})& = -1 & \mathrm{d}V^{\alpha\beta\gamma\delta}& = 4 \\
  \mathcal{W}(\mathcal{A}_\mu)& = 0 & \mathcal{W}(\mathcal{B}_{\mu}{}^{\lambda}{}_{\nu})& = 0 & \mathcal{W}(\Gamma_{\mu}{}^{\lambda}{}_{\nu})& = 0 & \mathrm{d}V^{\alpha\beta\gamma\delta}& = 1 
\end{align}
A generic term will have the following form
\begin{equation}
  \mathcal{O} = \mathcal{A}^m\mathcal{B}^n\Gamma^p\mathrm{d}V^q
\end{equation}
Applying the operators defined above, yield the equations
\begin{align}
  \mathcal{N}(\mathcal{O}) & = 4q - m - n - p & \mathcal{W}(\mathcal{O}) & = q 
\end{align}
Notice that we are interested in building scalar densities, meaning that the number of free index must be zero and the weight density must be equal to the unity. Therefore,
we have two constraints
\begin{align}
  m + n + p & = 4 & q & = 1
\end{align}
The terms contributing to the action are shown in the below table
\begin{table}[h]
\begin{center}
  \begin{tabular}{ | p{1cm} | p{1cm} | p{1cm} | p{3.5cm} | p{2cm} |}
  \hline
  $\mathcal{A}^m$ & $\mathcal{B}^n$ & $\Gamma^p$ & Type of configuration & Action term\\ \hline
  4 & 0 & 0 & $\mathcal{A}\mathcal{A}\mathcal{A}\mathcal{A}$ & 0 \\
  3 & 1 & 0 & $\mathcal{A}\mathcal{A}\mathcal{A}\mathcal{B}$ & 0 \\
  3 & 0 & 1 & $\mathcal{A}\mathcal{A}\mathcal{A}\nabla$ & 0  \\
  2 & 2 & 0 & $\mathcal{A}\mathcal{A}\mathcal{B}\mathcal{B}$ & $F_4$ \\
  2 & 1 & 1 & $\mathcal{A}\mathcal{A}\mathcal{B}\nabla$ & $D_6,D_7$ \\
  2 & 0 & 2 & $\mathcal{A}\mathcal{A}\nabla\nabla $ & 0    \\
  1 & 3 & 0 & $\mathcal{A}\mathcal{B}\mathcal{B}\mathcal{B}$ & $F_3$ \\
  1 & 2 & 1 & $\mathcal{A}\mathcal{B}\mathcal{B}\nabla$ & $D_4,D_5$ \\
  1 & 1 & 2 & $\mathcal{A}\mathcal{B}\nabla\nabla$ & $ B_3,B_4,B_5,E_2$\\
  1 & 0 & 3 & $\mathcal{A}\nabla\nabla\nabla$ & 0  \\
  0 & 4 & 0 & $\mathcal{B}\mathcal{B}\mathcal{B}\mathcal{B}$ & $F_1,F_2$ \\
  0 & 3 & 1 & $\mathcal{B}\mathcal{B}\mathcal{B}\nabla$ & $D_1,D_2,D_3$\\
  0 & 2 & 2 & $\mathcal{B}\mathcal{B}\nabla\nabla$ & $B_1,B_2,E_1$ \\
  0 & 1 & 3 & $\mathcal{B}\nabla\nabla\nabla$ & $C_1,C_2$ \\
  0 & 0 & 4 & $\nabla\nabla\nabla\nabla$ & 0\\ \hline
  \end{tabular}
  \caption{Possible terms contributing to the action of Polynomial Affine gravity}
\end{center}
\end{table}

From the above table, one use the symmetries of the tensor to see which terms will have non trivial contribution to the action. For example, the term with
four $\mathcal{A}$ does not contribute to the action since its contraction with the volume element is identically zero. Whenever two covariant
derivatives are contracted with the volume form they give a curvature tensor, and since the curvature is defined for the symmetric part of the connection, such 
curvature satisfy the torsion-free Bianchi identities, which relate some of the several possible contractions of the indices. An  additional argument that 
helps to drop contraction of indices is that $\mathcal{B}$ is traceless.

\section{Appendix B: Time coordinate reparametrization}

Under a coordinate transformation, the connection's coefficients changes as
\begin{equation}
  \frac{\partial^2 x^i}{\partial x^{'a}\partial x^{'b}} + \Gamma_{j}{}^{i}{}_{k}\frac{\partial x^j}{\partial x^{'a}}\frac{\partial x^k}{\partial x^{'b}} 
  = \Gamma'_{a}{}^{c}{}_{b}\frac{\partial x^i}{\partial x^{'c}}
\end{equation}
Next, consider a transformation of the form
\begin{align}
  t' & = t'(t) & r' & = r & \theta ' & = \theta & \phi' & = \phi
\end{align}
Then, we can find a \textit{coordinate system} where the connection's coefficients $\Gamma'_{0}{}^{0}{}_{0} = 0$ vanishes completely
\begin{equation}
  \frac{\partial^2 t}{\partial t^{'2}} + f \left(\frac{\partial t}{\partial t^{'}}\right)^2 = 0
\end{equation}
which can be written as a total derivative
\begin{equation}
  \frac{1}{X}\partial_{t'}\left(X \partial_{t'}t\right) = 0
\end{equation}
for $f = \frac{1}{X}\partial_t X$. From here, one have that
\begin{equation}
  t^{'} = \int \mathrm{d}t e^F(t)
\end{equation}
where $X = e^{\int \mathrm{d}t f(t)} = e^{F(t)}$. With the above transformation, it can be checked with ease that the effect of the time
reparametrisation is a scaling of the other time functions entering in the connection $g(t) \to g(t')$ and $h(t) \to h(t')$.

\section{Appendix C: Complete set of field equations}

\end{document}